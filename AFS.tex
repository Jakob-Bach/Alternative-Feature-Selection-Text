\documentclass{article}

\title{Alternative Feature Selection}
\author{Jakob Bach}

\usepackage[style=ieee, backend=bibtex]{biblatex}
\usepackage{amsmath} % mathematical symbols
\usepackage{amssymb} % mathematical symbols
\usepackage{enumitem} % nicely formatted enumerations
\usepackage{graphicx} % plots
\usepackage{subcaption} % figures with multiple sub-figures and sub-captions
\usepackage{hyperref} % links and URLs
\addbibresource{references.bib}

\begin{document}

\maketitle

\begin{abstract}
\end{abstract}

\section{Introduction}
\label{sec:introduction}

\paragraph{Motivation}

\paragraph{Problem statement}

\paragraph{Related work}

\paragraph{Contributions}

\paragraph{Results}

We publish our code\footnote{\url{https://github.com/Jakob-Bach/AFS}} and our experimental data\footnote{temporary link: \url{https://bwsyncandshare.kit.edu/s/xxx}; will be moved to a public repository after review}.

\paragraph{Outline}

Section~\ref{sec:related-work} reviews related work.
Section~\ref{sec:fundamentals} explains fundamentals of feature selection.
Section~\ref{sec:approach} introduces our approach for alternative feature sets.
Section~\ref{sec:experimental-design} describes our experimental design.
Section~\ref{sec:evaluation} presents the experimental results.
Section~\ref{sec:conclusion} concludes.

\section{Related Work}
\label{sec:related-work}

COALA \cite{bae2006coala}
Alternate Clustering \cite{bailey2014alternative}
alternative features for clustering \cite{tao2012novel}
features relevant global and in clusters \cite{guan2011unified}
integer problem to find alternative clusters \cite{bae2010clustering}
penalize overlap \cite{mueller2009relevant}

GMD \cite{trittenbach2019dimension}
Edouard \cite{fouche2021efficient}
combining subspaces \cite{nguyen20134s}

subgroup set discovery \cite{leeuwen2012diverse}

statistical equivalent subsets \cite{lagani2017feature, borboudakis2021extending, tsamardinos2003towards, dougherty2006number}
feature clustering \cite{mueller2021feature}
combining feature selection results \cite{woznica2012model}
genetic algo multi feature sets \cite{siddiqi2020genetic}

counterfactual surveys \cite{verma2020counterfactual, stepin2021survey}
MIP for counterfactuals \cite{mohammadi2021scaling}
SMT for counterfactuals \cite{karimi2020model}

constraint data mining \cite{grossi2017survey}

\section{Fundamentals}
\label{sec:fundamentals}

\subsection{Notation}

\subsection{Measuring Feature Quality}

filter, wrapper, embedded, importance/explanations

FS survey \cite{li2017feature, chandrashekar2014survey, guyon2003introduction}
interpretability \cite{carvalho2019machine}
SMT \cite{barrett2018satisfiability}
wrapper \cite{kohavi1997wrappers}

FCBF \cite{yu2003feature}
CFS \cite{hall1999correlation}
mRMR \cite{peng2005feature}
Relief \cite{kira1992feature}

can be combined with any feature importance score, e.g., also post-hoc local explanations like (absolute value of) LIME \cite{ribeiro2016should} or SHAP \cite{lundberg2017unified}

\section{Alternative Feature Selection (AFS) -- Approach}
\label{sec:approach}

\subsection{Quantifying Alternatives}

\subsection{Finding Alternatives}

can be combined with domain constraints \cite{groves2015toward}, group constraints \cite{yuan2006model}, cost constraints \cite{paclik2002feature}

focus on alternatives of sets as a whole - not 1-of-x for individual features (which requires user to define feature groups first, is similar to statistically equivalent signatures)

\section{Experimental Design}
\label{sec:experimental-design}

\subsection{Research Questions}

\subsection{Approaches}

feature selection
finding alternatives
prediction

\subsection{Evaluation Metrics}

\subsection{Datasets}

\subsection{Implementation}

We implement our experimental pipeline in Python.
For machine learning, we use \emph{scikit-learn}~\cite{pedregosa2011scikit-learn}.
To express and solve the constraint optimization problems, we use \emph{Z3}~\cite{deMoura2008z3}.

\section{Evaluation}
\label{sec:evaluation}

\section{Conclusions and Future Work}
\label{sec:conclusion}

\subsection{Conclusions}

\subsection{Future Work}

\printbibliography

\appendix

\section{Appendix}
\label{sec:appendix}

\end{document}
